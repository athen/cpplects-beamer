% 加载宏包
%===================注意======================%
% 在调用beamer.cls宏包后,以下宏包将自动调用,
% 不应单独调用这些宏包,以免发生冲突
% amsfonts, amsmath, amssymb, amsthm, 
% enumerate, geometry, graphics, graphicx, 
% hyperref, url, 
% ifpdf, keyval, xcolor, xxcolor
% =============================================%

% 由于setspace宏包会改变\@footnotetext,从面造成footcite不能引用的问题,
% 以下代码用于修正这一问题
\usepackage{etoolbox}

% ++++++++++++++++++++++++++++++++++++++++++++++++++
% 为一部分代码用于解决引用setspace调整间距宏包后造成的参考文献引用脚注
% 丢失问题
\makeatletter
% save the meaning of \@footnotetext
\let\BEAMER@footnotetext\@footnotetext
\makeatother
\usepackage{setspace} % 调整间距

%% 代码排版工具宏包
%% 需要预先安装 python 和 pygments。
%% 此宏包要加  -shell-escape 编译参数
%% 版本2.0,支持行内代码排版
\usepackage{minted}

% 绘制UML图
% 结pgf-umlcd宏包的关联关系命令进行了调整
\usepackage{pgf-umlcd-gn}

% 添加附加文件
\usepackage[author=耿楠,
 scale=0.6,
 color=blue,
 mimetype=text/plain,
 subject=源代码,
 description=打开或下载该源代码,
 icon=Paperclip]{attachfile2}

%% 根据章节需要加载其它不同的宏包
\ifcase\chno
% 第0章
% 图标字体
\usepackage{fontawesome5}

\or% 第1章
% 绘制UML图
%\usepackage{pgf-umlcd}

% 绘制开发流程需要的宏包,应该必为TikZ绘制(2020.02.08)
% 绘制阴影
\usepackage{pst-blur}%
% pstricks用于绘制流程图,但与menus宏冲突
\usepackage{pstricks}%
% pstricks扩展包
\usepackage{pstricks-add}%

\or% 第2章
% 绘制UML图
%\usepackage{pgf-umlcd}
% 多行合并表格包
\usepackage{multirow}
% 绘制内存图
\usepackage{bytefield}
% 三线表格
\usepackage{booktabs}

\or% 第3章
% 彩色文本框
\usepackage{tcolorbox}
% % =========解决minted包排版代码的跨页问题=========
% \usepackage[linecolor=black, topline=true, bottomline=true,
% leftline=false, rightline=false,
% backgroundcolor=yellow!20!white]{mdframed}
% ========排版键盘组合和菜单的宏包=========
\usepackage[os=win]{menukeys}
% 插图重叠
\usepackage[abs]{overpic}
% 一些图标等特殊符号
\usepackage{pifont}
% 绘制内存图
\usepackage{bytefield}
% 流程图绘制宏包
\usepackage{tikz-flowchart}

\or% 第4章
% 绘制UML图
%\usepackage{pgf-umlcd}
% 带边框的小页环境
\usepackage{boxedminipage}
\usepackage{graphbox}%

\or% 第5章
% 绘制UML图
%\usepackage{pgf-umlcd}
% 一些图标等特殊符号
\usepackage{pifont}
% 斜线表头包
\usepackage{diagbox}
% 表格tabular背景色包
\usepackage{colortbl}

\or% 第6章
% 绘制UML图
%\usepackage{pgf-umlcd}

% 一些图标等特殊符号
\usepackage{pifont}

\or% 第7章
% 绘制UML图
% \usepackage{pgf-umlcd}

\or% 第8章
% 绘制UML图
% \usepackage{pgf-umlcd}

\or% 第9章
% 绘制UML图
% \usepackage{pgf-umlcd}

\or% 第10章
% 绘制UML图
% \usepackage{pgf-umlcd}

\fi

%\usepackage{tcolorbox}
 
%%% Local Variables: 
%%% mode: latex
%%% TeX-master: "../main.tex"
%%% End:

